\chapter{Empirische Ergebnisse und Evaluierung} \label{cha:empiricalPart}
Die Durchführung der empirischen Untersuchung ist nachvollziehbar zu dokumentieren.
Die empirischen Ergebnisse der Arbeit sowie auch die dabei aufgetretenen Probleme und deren Behandlung. sind in übersichtlicher Form darzustellen. Die gewählte Form der Darstellung ist vom gewählten Datenmaterial und den in der Einleitung gesetzten Zielen abhängig.
Wichtig ist die gedanklich klare Unterscheidung zwischen der Darstellung der Ergebnisse (Kapitel 4) und der Interpretation/Bewertung der Ergebnisse (Kapitel 5).
Unterkapitel sind sinnvoll zu wählen.



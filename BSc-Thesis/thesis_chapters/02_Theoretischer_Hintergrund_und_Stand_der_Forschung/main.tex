

\chapter{Theoretischer Hintergrund und Stand der Forschung} \label{cha:fundaments}
Die in dieser Formatvorlage beispielhaft enthaltenen Überschriften sind auf die im konkreten Fall tatsächlich passenden Überschriften anzupassen.\\
In diesem Teil der Arbeit werden die zum eindeutigen Verständnis unbedingt erforderlichen 
Grundlagen und Definitionen sowie die Erklärung wichtiger Begriffe angeführt.\\
Die Gliederungspunkte müssen möglichst prägnant bezeichnet werden.



\section{Stand des Wissens/Forschung/Technik} \label{cha:hydrostaticPressure}
Die neuesten Entwicklungen und Arbeiten auf diesem Gebiet (Stand der Wissenschaft oder auch state-of-the-art) 
sind darzulegen, wobei diese je nach Thema auch in der 3. Gliederungsebene behandelt werden können.\\
Dieses Unterkapitel inkl. eventueller Unterkapitel muss enthalten sein.



\section{Weitere frei zu wählende Unterkapitel} \label{cha:generalDefinitions}


\section{Zwischenfazit} \label{cha:zwischenfazit}
Alternativ kann dieses Unterkapitel auch „Interpretation in Bezug auf die Forschungsfrage und Literatur“ genannt werden.
Sie fassen die Erkenntnisse kurz zusammen und beantworten bzw. interpretieren diese im Sinne der Forschungsfrage.
Dieses Unterkapitel muss enthalten sein.



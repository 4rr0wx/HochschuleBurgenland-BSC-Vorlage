\chapter{Empirischer Teil} \label{cha:empiricalPart}
Die Durchführung der empirischen Untersuchung ist nachvollziehbar zu dokumentieren sowie auch die dabei aufgetretenen Probleme und deren Behandlung. Der Umfang ergibt sich aus der Art der Bearbeitung.

\section{Beispiel für eine Tabelle}

Tabelle \ref{tab:example} zeigt ein Beispiel für eine Tabelle.

\begin{table}[h]
    \centering
    \renewcommand{\arraystretch}{1.2}
    \begin{tabular}{|c|c|c|c|}
        \hline
        \textbf{Größe A} & \textbf{Größe B} & \textbf{Größe C} & \textbf{Größe D} \\
        \textbf{Einheit A} & \textbf{Einheit B} & \textbf{Einheit C} & \textbf{Einheit D} \\
        \hline
        50  & 0,16 & 0,50 & 0,10 \\
        100 & 0,31 & 0,96 & 0,26 \\
        200 & 0,57 & 1,45 & 0,50 \\
        300 & 0,95 & 2,14 & 0,75 \\
        500 & 1,45 & 2,74 & 1,43 \\
        1000 & 2,14 & 3,05 & 1,43 \\
        1500 & 2,57 & 3,75 & 1,43 \\
        \hline
    \end{tabular}
    \caption{Beispiel für eine Tabelle}
    \label{tab:example}
\end{table}

\section{Beispiel für eine Abbildung}

Abbildung \ref{fig:example} zeigt ein Beispiel für eine Abbildung oder Grafik.

\begin{figure}[h]
    \centering
    \includegraphics[width=0.8\textwidth]{figures/Hochschule-Burgenland-Logo.jpg} 
    \caption{Beispiel für eine Abbildung}
    \label{fig:example}
\end{figure}

\section{Mathematische Herleitung}

Mathematisch werden die Zusammenhänge durch folgende Formel beschrieben:

\begin{equation}
    (x + a)^n = \sum_{k=0}^{n} \binom{n}{k} a^k x^{n-k}
\end{equation}

\textbf{Formel 1:} Mathematische Herleitung

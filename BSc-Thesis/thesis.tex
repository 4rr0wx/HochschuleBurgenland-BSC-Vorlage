\documentclass[twoside=off,toc=chapterentrywithdots,a4paper,fontsize=12pt,listof=totoc,bibliography=totoc]{scrbook}
\input{00_preamble/german}
\documentclass[twoside=off,toc=chapterentrywithdots,a4paper,fontsize=12pt,listof=totoc,bibliography=totoc]{scrbook}
\input{00_preamble/german}
\documentclass[twoside=off,toc=chapterentrywithdots,a4paper,fontsize=12pt,listof=totoc,bibliography=totoc]{scrbook}
\input{00_preamble/german}
\documentclass[twoside=off,toc=chapterentrywithdots,a4paper,fontsize=12pt,listof=totoc,bibliography=totoc]{scrbook}
\input{00_preamble/german}
\input{00_preamble/thesis}
\input{01_data/yourData}


\hypersetup{
    pdftitle={\yourThesisTitle},
    pdfsubject={\typeOfWork},
    pdfauthor={\yourNameInclTitle},
    breaklinks,               % permits line breaks for long links
    bookmarksnumbered,        % ... and include section numbers
    linktocpage,              % "make page number, not text, be link on TOC ..."
    colorlinks,               % yes ...
    linkcolor=black,          % normal internal links;
    anchorcolor=black,        % don't make scientific papers too much colourful => "black"
    citecolor=black,
    urlcolor=blue,            % quite common
    pdfstartview={Fit},       % "Fit" fits the page to the window
    pdfpagemode=UseOutlines,  % open bookmarks in Acrobat
    plainpages=false          % avoids duplicate page number problem
}

% important update for glossaries, before document 
%\makeglossaries % see https://www.overleaf.com/learn/latex/glossaries
%\loadglsentries{C_BackMatter/defns} 

%\GlsXtrLoadResources[src={C_BackMatter/defns}]
% adjust indexgroup style:
\renewcommand{\glstreenamefmt}[1]{#1}
\renewcommand{\glstreegroupheaderfmt}[1]{\textbf{#1}}

\newcommand{\tabitem}{~~\llap{\textbullet}~~} %bullet points in table

%remove all the things you don't need and add all the things you need. you can also change the order of the elements at your will.
\begin{document}
%\makeindex % create an index

\input{A_FrontMatter/titlepage} %Creates the titlepage
\frontmatter{} % turns off chapter numbering and uses roman numerals for page numbers

\input{thesis_chapters/00_Vorwort.tex}     % adds the acknowledgements

%Inhaltsverzeichnis folgender Block \renewcommand{\contentsname}{Inhalt} \tableofcontents{}
\renewcommand{\contentsname}{Inhaltsverzeichnis}
\tableofcontents{}

%Folgender Block fügt eine leere Seite ohne Nummerierung hinzu 
\newpage
\pagestyle{plain} % oder plain für normale Nummerierung beibehalten
\null 
\newpage

\renewcommand{\figurename}{Abbildung}

%\input{A_FrontMatter/kurzfassung} % adds the "Kurzfassung" in german.
%\input{A_FrontMatter/abstracts} % adds the abstract
%\pagebreak

\input{thesis_chapters/00_abstract_de}
\input{thesis_chapters/00_abstract_en}
\microtypesetup{protrusion=false}


\mainmatter{} % turns on chapter numbering, resets page numbering and uses arabic numerals for page numbers
\input{thesis_chapters/01_Einleitung/_main.tex}
\input{thesis_chapters/02_Theoretischer_Hintergrund_und_Stand_der_Forschung/main.tex}
\input{thesis_chapters/03_Methoden_und_Forschungsdesign/_main.tex}
\input{thesis_chapters/04_Empirische_Ergebnisse_und_Evaluierung/_main.tex}
\input{thesis_chapters/05_Diskussion_und_Interpretation/_main.tex}
\input{thesis_chapters/06_Zusammenfassung/_main.tex}

%\nocite{*} % includes complete bibliography, including entries that remain uncited, RMF
\printbibliography[heading=bibnumbered, title=\literatureLabel] % bibliography


% Use an optional list of tables / figures / algorithms / listings(code).

\renewcommand{\listfigurename}{Abbildungsverzeichnis}
\listoffigures
\clearpage

\renewcommand{\listtablename}{Tabellenverzeichnis}
\listoftables
\clearpage

%\lstlistoflistings
%\cleardoublepage{}
%\listofalgorithms
%\addcontentsline{toc}{chapter}{List of Algorithms} 
%\cleardoublepage{}
%\cleardoublepage{}
%\microtypesetup{protrusion=true}

%\printglossary[type=\acronymtype]
%\printglossary

%\printunsrtglossaries

%\printunsrtglossary[type=\acronymtype, style=tab_style, nonumberlist=true, title={Acronyms}]
%\printunsrtglossary[type=entry, style=tab_style_sym, title={Definitions}]

%\cleardoublepage{}
%\pagebreak

%add your end matter here
%\{} % resets chapter numbering, uses letters for chapter numbers and doesn't fiddle with page numbering
%\backmatter{} %turns off chapter numbering and doesn't fiddle with page numberin
%\input{C_BackMatter/glossary}
\input{Formelverzeichnis.tex}
\input{acronyms}
\clearpage
\input{C_BackMatter/Anhang/_main}
\clearpage
\input{Tooldokumentation.tex}
\input{C_BackMatter/Eidesstattliche Erklärung.tex}

%\input{C_BackMatter/Anhang}
\end{document}
% if not using overleaf - data is needed separately
\newcommand{\typeOfWork}{Bachelorarbeit}
\newcommand{\titleToObtain}{Bachelor of Science in Engineering}
\newcommand{\studyProgram}{Bachelorstudiengang IT Infrastruktur-Management}
\newcommand{\yourNameInclTitle}{Max Mustermann}
\newcommand{\yourMatNumber}{2210781666}
\newcommand{\supervisorNameInclTitle}{Univ.-Prof. Daisy Musterfrau}
\newcommand{\departmentName}{Department Informationstechnologie}
\newcommand{\yourThesisTitle}{Beispieltitel des Bachelorarbeits-Templates}
\newcommand{\thesisDate}{\ordinalnum{27} Juli 2024}
\newcommand{\universityCityCountry}{Hochschule-Burgenland, Eisenstadt, Österreich}
\newcommand{\literatureLabel}{Literaturverzeichnis}



\hypersetup{
    pdftitle={\yourThesisTitle},
    pdfsubject={\typeOfWork},
    pdfauthor={\yourNameInclTitle},
    breaklinks,               % permits line breaks for long links
    bookmarksnumbered,        % ... and include section numbers
    linktocpage,              % "make page number, not text, be link on TOC ..."
    colorlinks,               % yes ...
    linkcolor=black,          % normal internal links;
    anchorcolor=black,        % don't make scientific papers too much colourful => "black"
    citecolor=black,
    urlcolor=blue,            % quite common
    pdfstartview={Fit},       % "Fit" fits the page to the window
    pdfpagemode=UseOutlines,  % open bookmarks in Acrobat
    plainpages=false          % avoids duplicate page number problem
}

% important update for glossaries, before document 
%\makeglossaries % see https://www.overleaf.com/learn/latex/glossaries
%\loadglsentries{C_BackMatter/defns} 

%\GlsXtrLoadResources[src={C_BackMatter/defns}]
% adjust indexgroup style:
\renewcommand{\glstreenamefmt}[1]{#1}
\renewcommand{\glstreegroupheaderfmt}[1]{\textbf{#1}}

\newcommand{\tabitem}{~~\llap{\textbullet}~~} %bullet points in table

%remove all the things you don't need and add all the things you need. you can also change the order of the elements at your will.
\begin{document}
%\makeindex % create an index

\input{A_FrontMatter/titlepage} %Creates the titlepage
\frontmatter{} % turns off chapter numbering and uses roman numerals for page numbers



\thispagestyle{plain}
\chapter{Vorwort}

Hier ist der Platz für persönliche Worte wie zum Beispiel Dankesworte an Betreuer, Berater, Freunde und Familie.\\
Vorname Familienname
Eisenstadt, 27.November 2024


\vspace{2cm}


\begin{flushleft}
    Max Mustermann
\end{flushleft}
Eisenstadt, \thesisDate

     % adds the acknowledgements

%Inhaltsverzeichnis folgender Block \renewcommand{\contentsname}{Inhalt} \tableofcontents{}
\renewcommand{\contentsname}{Inhaltsverzeichnis}
\tableofcontents{}

%Folgender Block fügt eine leere Seite ohne Nummerierung hinzu 
\newpage
\pagestyle{plain} % oder plain für normale Nummerierung beibehalten
\null 
\newpage

\renewcommand{\figurename}{Abbildung}

%\input{A_FrontMatter/kurzfassung} % adds the "Kurzfassung" in german.
%\input{A_FrontMatter/abstracts} % adds the abstract
%\pagebreak

\input{thesis_chapters/00_abstract_de}

\chapter{Abstract}

Komprimierter Inhalt der Arbeit (Kurzfassung) in Englisch.\\
Länge maximal 1 Seite

\autocite[S.24]{ederer2024}


\microtypesetup{protrusion=false}


\mainmatter{} % turns on chapter numbering, resets page numbering and uses arabic numerals for page numbers


\appendix
\renewcommand{\thechapter}{\Alph{chapter}}
\renewcommand{\thesection}{\Alph{section}}

\phantomsection
\addcontentsline{toc}{chapter}{Anhang}
\chapter*{Anhang}
Im Anhang sind all jene Informationen zu finden, die den Lesefluss der Arbeit stören würden, doch für die 
Vollständigkeit der Arbeit notwendig sind, z.B. Datentabellen, Fragebögen, etc.
Der Anhang unterliegt nicht mehr der Gliederung, sondern wird mit Anhang A, Anhang B, … gekennzeichnet.


\chapter{Theoretischer Hintergrund und Stand der Forschung} \label{cha:fundaments}
Die in dieser Formatvorlage beispielhaft enthaltenen Überschriften sind auf die im konkreten Fall tatsächlich passenden Überschriften anzupassen.\\
In diesem Teil der Arbeit werden die zum eindeutigen Verständnis unbedingt erforderlichen 
Grundlagen und Definitionen sowie die Erklärung wichtiger Begriffe angeführt.\\
Die Gliederungspunkte müssen möglichst prägnant bezeichnet werden.



\input{thesis_chapters/02_Theoretischer_Hintergrund_und_Stand_der_Forschung/01_Stand_des_Wissens.tex}
\input{thesis_chapters/02_Theoretischer_Hintergrund_und_Stand_der_Forschung/02_Weitere frei zu wählende Unterkapitel.tex}
\input{thesis_chapters/02_Theoretischer_Hintergrund_und_Stand_der_Forschung/03_Zwischenfazit.tex}



\appendix
\renewcommand{\thechapter}{\Alph{chapter}}
\renewcommand{\thesection}{\Alph{section}}

\phantomsection
\addcontentsline{toc}{chapter}{Anhang}
\chapter*{Anhang}
Im Anhang sind all jene Informationen zu finden, die den Lesefluss der Arbeit stören würden, doch für die 
Vollständigkeit der Arbeit notwendig sind, z.B. Datentabellen, Fragebögen, etc.
Der Anhang unterliegt nicht mehr der Gliederung, sondern wird mit Anhang A, Anhang B, … gekennzeichnet.


\appendix
\renewcommand{\thechapter}{\Alph{chapter}}
\renewcommand{\thesection}{\Alph{section}}

\phantomsection
\addcontentsline{toc}{chapter}{Anhang}
\chapter*{Anhang}
Im Anhang sind all jene Informationen zu finden, die den Lesefluss der Arbeit stören würden, doch für die 
Vollständigkeit der Arbeit notwendig sind, z.B. Datentabellen, Fragebögen, etc.
Der Anhang unterliegt nicht mehr der Gliederung, sondern wird mit Anhang A, Anhang B, … gekennzeichnet.


\appendix
\renewcommand{\thechapter}{\Alph{chapter}}
\renewcommand{\thesection}{\Alph{section}}

\phantomsection
\addcontentsline{toc}{chapter}{Anhang}
\chapter*{Anhang}
Im Anhang sind all jene Informationen zu finden, die den Lesefluss der Arbeit stören würden, doch für die 
Vollständigkeit der Arbeit notwendig sind, z.B. Datentabellen, Fragebögen, etc.
Der Anhang unterliegt nicht mehr der Gliederung, sondern wird mit Anhang A, Anhang B, … gekennzeichnet.


\appendix
\renewcommand{\thechapter}{\Alph{chapter}}
\renewcommand{\thesection}{\Alph{section}}

\phantomsection
\addcontentsline{toc}{chapter}{Anhang}
\chapter*{Anhang}
Im Anhang sind all jene Informationen zu finden, die den Lesefluss der Arbeit stören würden, doch für die 
Vollständigkeit der Arbeit notwendig sind, z.B. Datentabellen, Fragebögen, etc.
Der Anhang unterliegt nicht mehr der Gliederung, sondern wird mit Anhang A, Anhang B, … gekennzeichnet.

%\nocite{*} % includes complete bibliography, including entries that remain uncited, RMF
\printbibliography[heading=bibnumbered, title=\literatureLabel] % bibliography


% Use an optional list of tables / figures / algorithms / listings(code).

\renewcommand{\listfigurename}{Abbildungsverzeichnis}
\listoffigures
\clearpage

\renewcommand{\listtablename}{Tabellenverzeichnis}
\listoftables
\clearpage

%\lstlistoflistings
%\cleardoublepage{}
%\listofalgorithms
%\addcontentsline{toc}{chapter}{List of Algorithms} 
%\cleardoublepage{}
%\cleardoublepage{}
%\microtypesetup{protrusion=true}

%\printglossary[type=\acronymtype]
%\printglossary

%\printunsrtglossaries

%\printunsrtglossary[type=\acronymtype, style=tab_style, nonumberlist=true, title={Acronyms}]
%\printunsrtglossary[type=entry, style=tab_style_sym, title={Definitions}]

%\cleardoublepage{}
%\pagebreak

%add your end matter here
%\{} % resets chapter numbering, uses letters for chapter numbers and doesn't fiddle with page numbering
%\backmatter{} %turns off chapter numbering and doesn't fiddle with page numberin
%\input{C_BackMatter/glossary}
\input{Formelverzeichnis.tex}
\input{acronyms}
\clearpage


\appendix
\renewcommand{\thechapter}{\Alph{chapter}}
\renewcommand{\thesection}{\Alph{section}}

\phantomsection
\addcontentsline{toc}{chapter}{Anhang}
\chapter*{Anhang}
Im Anhang sind all jene Informationen zu finden, die den Lesefluss der Arbeit stören würden, doch für die 
Vollständigkeit der Arbeit notwendig sind, z.B. Datentabellen, Fragebögen, etc.
Der Anhang unterliegt nicht mehr der Gliederung, sondern wird mit Anhang A, Anhang B, … gekennzeichnet.
\clearpage
\phantomsection
\addcontentsline{toc}{chapter}{Tooldokumentation}
\chapter*{Tooldokumentation}
Hier sind alle genutzten (KI) Tools zu dokumentieren, die über die im Methodenkapitel eingesetzten Tools 
hinausgehen.\\
\begin{table}[h]
    \centering
    \renewcommand{\arraystretch}{1.2}
    \begin{tabular}{|c|c|c|c|}
        \hline
        \textbf{(KI) Tool} & \textbf{Anwendungsbereich} & \textbf{Betroffener Teil} & \textbf{Anmerkungen} \\
        \hline
        text  & text & text & text \\
        text & text & text& text \\
        \hline
    \end{tabular}
\end{table}
\input{C_BackMatter/Eidesstattliche Erklärung.tex}

%\input{C_BackMatter/Anhang}
\end{document}
% if not using overleaf - data is needed separately
\newcommand{\typeOfWork}{Bachelorarbeit}
\newcommand{\titleToObtain}{Bachelor of Science in Engineering}
\newcommand{\studyProgram}{Bachelorstudiengang IT Infrastruktur-Management}
\newcommand{\yourNameInclTitle}{Max Mustermann}
\newcommand{\yourMatNumber}{2210781666}
\newcommand{\supervisorNameInclTitle}{Univ.-Prof. Daisy Musterfrau}
\newcommand{\departmentName}{Department Informationstechnologie}
\newcommand{\yourThesisTitle}{Beispieltitel des Bachelorarbeits-Templates}
\newcommand{\thesisDate}{\ordinalnum{27} Juli 2024}
\newcommand{\universityCityCountry}{Hochschule-Burgenland, Eisenstadt, Österreich}
\newcommand{\literatureLabel}{Literaturverzeichnis}



\hypersetup{
    pdftitle={\yourThesisTitle},
    pdfsubject={\typeOfWork},
    pdfauthor={\yourNameInclTitle},
    breaklinks,               % permits line breaks for long links
    bookmarksnumbered,        % ... and include section numbers
    linktocpage,              % "make page number, not text, be link on TOC ..."
    colorlinks,               % yes ...
    linkcolor=black,          % normal internal links;
    anchorcolor=black,        % don't make scientific papers too much colourful => "black"
    citecolor=black,
    urlcolor=blue,            % quite common
    pdfstartview={Fit},       % "Fit" fits the page to the window
    pdfpagemode=UseOutlines,  % open bookmarks in Acrobat
    plainpages=false          % avoids duplicate page number problem
}

% important update for glossaries, before document 
%\makeglossaries % see https://www.overleaf.com/learn/latex/glossaries
%\loadglsentries{C_BackMatter/defns} 

%\GlsXtrLoadResources[src={C_BackMatter/defns}]
% adjust indexgroup style:
\renewcommand{\glstreenamefmt}[1]{#1}
\renewcommand{\glstreegroupheaderfmt}[1]{\textbf{#1}}

\newcommand{\tabitem}{~~\llap{\textbullet}~~} %bullet points in table

%remove all the things you don't need and add all the things you need. you can also change the order of the elements at your will.
\begin{document}
%\makeindex % create an index

\input{A_FrontMatter/titlepage} %Creates the titlepage
\frontmatter{} % turns off chapter numbering and uses roman numerals for page numbers



\thispagestyle{plain}
\chapter{Vorwort}

Hier ist der Platz für persönliche Worte wie zum Beispiel Dankesworte an Betreuer, Berater, Freunde und Familie.\\
Vorname Familienname
Eisenstadt, 27.November 2024


\vspace{2cm}


\begin{flushleft}
    Max Mustermann
\end{flushleft}
Eisenstadt, \thesisDate

     % adds the acknowledgements

%Inhaltsverzeichnis folgender Block \renewcommand{\contentsname}{Inhalt} \tableofcontents{}
\renewcommand{\contentsname}{Inhaltsverzeichnis}
\tableofcontents{}

%Folgender Block fügt eine leere Seite ohne Nummerierung hinzu 
\newpage
\pagestyle{plain} % oder plain für normale Nummerierung beibehalten
\null 
\newpage

\renewcommand{\figurename}{Abbildung}

%\input{A_FrontMatter/kurzfassung} % adds the "Kurzfassung" in german.
%\input{A_FrontMatter/abstracts} % adds the abstract
%\pagebreak

\input{thesis_chapters/00_abstract_de}

\chapter{Abstract}

Komprimierter Inhalt der Arbeit (Kurzfassung) in Englisch.\\
Länge maximal 1 Seite

\autocite[S.24]{ederer2024}


\microtypesetup{protrusion=false}


\mainmatter{} % turns on chapter numbering, resets page numbering and uses arabic numerals for page numbers


\appendix
\renewcommand{\thechapter}{\Alph{chapter}}
\renewcommand{\thesection}{\Alph{section}}

\phantomsection
\addcontentsline{toc}{chapter}{Anhang}
\chapter*{Anhang}
Im Anhang sind all jene Informationen zu finden, die den Lesefluss der Arbeit stören würden, doch für die 
Vollständigkeit der Arbeit notwendig sind, z.B. Datentabellen, Fragebögen, etc.
Der Anhang unterliegt nicht mehr der Gliederung, sondern wird mit Anhang A, Anhang B, … gekennzeichnet.


\chapter{Theoretischer Hintergrund und Stand der Forschung} \label{cha:fundaments}
Die in dieser Formatvorlage beispielhaft enthaltenen Überschriften sind auf die im konkreten Fall tatsächlich passenden Überschriften anzupassen.\\
In diesem Teil der Arbeit werden die zum eindeutigen Verständnis unbedingt erforderlichen 
Grundlagen und Definitionen sowie die Erklärung wichtiger Begriffe angeführt.\\
Die Gliederungspunkte müssen möglichst prägnant bezeichnet werden.



\section{Stand des Wissens/Forschung/Technik} \label{cha:hydrostaticPressure}
Die neuesten Entwicklungen und Arbeiten auf diesem Gebiet (Stand der Wissenschaft oder auch state-of-the-art) 
sind darzulegen, wobei diese je nach Thema auch in der 3. Gliederungsebene behandelt werden können.\\
Dieses Unterkapitel inkl. eventueller Unterkapitel muss enthalten sein.



\section{Weitere frei zu wählende Unterkapitel} \label{cha:generalDefinitions}


\section{Zwischenfazit} \label{cha:zwischenfazit}
Alternativ kann dieses Unterkapitel auch „Interpretation in Bezug auf die Forschungsfrage und Literatur“ genannt werden.
Sie fassen die Erkenntnisse kurz zusammen und beantworten bzw. interpretieren diese im Sinne der Forschungsfrage.
Dieses Unterkapitel muss enthalten sein.





\appendix
\renewcommand{\thechapter}{\Alph{chapter}}
\renewcommand{\thesection}{\Alph{section}}

\phantomsection
\addcontentsline{toc}{chapter}{Anhang}
\chapter*{Anhang}
Im Anhang sind all jene Informationen zu finden, die den Lesefluss der Arbeit stören würden, doch für die 
Vollständigkeit der Arbeit notwendig sind, z.B. Datentabellen, Fragebögen, etc.
Der Anhang unterliegt nicht mehr der Gliederung, sondern wird mit Anhang A, Anhang B, … gekennzeichnet.


\appendix
\renewcommand{\thechapter}{\Alph{chapter}}
\renewcommand{\thesection}{\Alph{section}}

\phantomsection
\addcontentsline{toc}{chapter}{Anhang}
\chapter*{Anhang}
Im Anhang sind all jene Informationen zu finden, die den Lesefluss der Arbeit stören würden, doch für die 
Vollständigkeit der Arbeit notwendig sind, z.B. Datentabellen, Fragebögen, etc.
Der Anhang unterliegt nicht mehr der Gliederung, sondern wird mit Anhang A, Anhang B, … gekennzeichnet.


\appendix
\renewcommand{\thechapter}{\Alph{chapter}}
\renewcommand{\thesection}{\Alph{section}}

\phantomsection
\addcontentsline{toc}{chapter}{Anhang}
\chapter*{Anhang}
Im Anhang sind all jene Informationen zu finden, die den Lesefluss der Arbeit stören würden, doch für die 
Vollständigkeit der Arbeit notwendig sind, z.B. Datentabellen, Fragebögen, etc.
Der Anhang unterliegt nicht mehr der Gliederung, sondern wird mit Anhang A, Anhang B, … gekennzeichnet.


\appendix
\renewcommand{\thechapter}{\Alph{chapter}}
\renewcommand{\thesection}{\Alph{section}}

\phantomsection
\addcontentsline{toc}{chapter}{Anhang}
\chapter*{Anhang}
Im Anhang sind all jene Informationen zu finden, die den Lesefluss der Arbeit stören würden, doch für die 
Vollständigkeit der Arbeit notwendig sind, z.B. Datentabellen, Fragebögen, etc.
Der Anhang unterliegt nicht mehr der Gliederung, sondern wird mit Anhang A, Anhang B, … gekennzeichnet.

%\nocite{*} % includes complete bibliography, including entries that remain uncited, RMF
\printbibliography[heading=bibnumbered, title=\literatureLabel] % bibliography


% Use an optional list of tables / figures / algorithms / listings(code).

\renewcommand{\listfigurename}{Abbildungsverzeichnis}
\listoffigures
\clearpage

\renewcommand{\listtablename}{Tabellenverzeichnis}
\listoftables
\clearpage

%\lstlistoflistings
%\cleardoublepage{}
%\listofalgorithms
%\addcontentsline{toc}{chapter}{List of Algorithms} 
%\cleardoublepage{}
%\cleardoublepage{}
%\microtypesetup{protrusion=true}

%\printglossary[type=\acronymtype]
%\printglossary

%\printunsrtglossaries

%\printunsrtglossary[type=\acronymtype, style=tab_style, nonumberlist=true, title={Acronyms}]
%\printunsrtglossary[type=entry, style=tab_style_sym, title={Definitions}]

%\cleardoublepage{}
%\pagebreak

%add your end matter here
%\{} % resets chapter numbering, uses letters for chapter numbers and doesn't fiddle with page numbering
%\backmatter{} %turns off chapter numbering and doesn't fiddle with page numberin
%\input{C_BackMatter/glossary}
\input{Formelverzeichnis.tex}
\input{acronyms}
\clearpage


\appendix
\renewcommand{\thechapter}{\Alph{chapter}}
\renewcommand{\thesection}{\Alph{section}}

\phantomsection
\addcontentsline{toc}{chapter}{Anhang}
\chapter*{Anhang}
Im Anhang sind all jene Informationen zu finden, die den Lesefluss der Arbeit stören würden, doch für die 
Vollständigkeit der Arbeit notwendig sind, z.B. Datentabellen, Fragebögen, etc.
Der Anhang unterliegt nicht mehr der Gliederung, sondern wird mit Anhang A, Anhang B, … gekennzeichnet.
\clearpage
\phantomsection
\addcontentsline{toc}{chapter}{Tooldokumentation}
\chapter*{Tooldokumentation}
Hier sind alle genutzten (KI) Tools zu dokumentieren, die über die im Methodenkapitel eingesetzten Tools 
hinausgehen.\\
\begin{table}[h]
    \centering
    \renewcommand{\arraystretch}{1.2}
    \begin{tabular}{|c|c|c|c|}
        \hline
        \textbf{(KI) Tool} & \textbf{Anwendungsbereich} & \textbf{Betroffener Teil} & \textbf{Anmerkungen} \\
        \hline
        text  & text & text & text \\
        text & text & text& text \\
        \hline
    \end{tabular}
\end{table}
\input{C_BackMatter/Eidesstattliche Erklärung.tex}

%\input{C_BackMatter/Anhang}
\end{document}
% if not using overleaf - data is needed separately
\newcommand{\typeOfWork}{Bachelorarbeit}
\newcommand{\titleToObtain}{Bachelor of Science in Engineering}
\newcommand{\studyProgram}{Bachelorstudiengang IT Infrastruktur-Management}
\newcommand{\yourNameInclTitle}{Max Mustermann}
\newcommand{\yourMatNumber}{2210781666}
\newcommand{\supervisorNameInclTitle}{Univ.-Prof. Daisy Musterfrau}
\newcommand{\departmentName}{Department Informationstechnologie}
\newcommand{\yourThesisTitle}{Beispieltitel des Bachelorarbeits-Templates}
\newcommand{\thesisDate}{\ordinalnum{27} Juli 2024}
\newcommand{\universityCityCountry}{Hochschule-Burgenland, Eisenstadt, Österreich}
\newcommand{\literatureLabel}{Literaturverzeichnis}



\hypersetup{
    pdftitle={\yourThesisTitle},
    pdfsubject={\typeOfWork},
    pdfauthor={\yourNameInclTitle},
    breaklinks,               % permits line breaks for long links
    bookmarksnumbered,        % ... and include section numbers
    linktocpage,              % "make page number, not text, be link on TOC ..."
    colorlinks,               % yes ...
    linkcolor=black,          % normal internal links;
    anchorcolor=black,        % don't make scientific papers too much colourful => "black"
    citecolor=black,
    urlcolor=blue,            % quite common
    pdfstartview={Fit},       % "Fit" fits the page to the window
    pdfpagemode=UseOutlines,  % open bookmarks in Acrobat
    plainpages=false          % avoids duplicate page number problem
}

% important update for glossaries, before document 
%\makeglossaries % see https://www.overleaf.com/learn/latex/glossaries
%\loadglsentries{C_BackMatter/defns} 

%\GlsXtrLoadResources[src={C_BackMatter/defns}]
% adjust indexgroup style:
\renewcommand{\glstreenamefmt}[1]{#1}
\renewcommand{\glstreegroupheaderfmt}[1]{\textbf{#1}}

\newcommand{\tabitem}{~~\llap{\textbullet}~~} %bullet points in table

%remove all the things you don't need and add all the things you need. you can also change the order of the elements at your will.
\begin{document}
%\makeindex % create an index

\input{A_FrontMatter/titlepage} %Creates the titlepage
\frontmatter{} % turns off chapter numbering and uses roman numerals for page numbers



\thispagestyle{plain}
\chapter{Vorwort}

Hier ist der Platz für persönliche Worte wie zum Beispiel Dankesworte an Betreuer, Berater, Freunde und Familie.\\
Vorname Familienname
Eisenstadt, 27.November 2024


\vspace{2cm}


\begin{flushleft}
    Max Mustermann
\end{flushleft}
Eisenstadt, \thesisDate

     % adds the acknowledgements

%Inhaltsverzeichnis folgender Block \renewcommand{\contentsname}{Inhalt} \tableofcontents{}
\renewcommand{\contentsname}{Inhaltsverzeichnis}
\tableofcontents{}

%Folgender Block fügt eine leere Seite ohne Nummerierung hinzu 
\newpage
\pagestyle{plain} % oder plain für normale Nummerierung beibehalten
\null 
\newpage

\renewcommand{\figurename}{Abbildung}

%\input{A_FrontMatter/kurzfassung} % adds the "Kurzfassung" in german.
%\input{A_FrontMatter/abstracts} % adds the abstract
%\pagebreak

\input{thesis_chapters/00_abstract_de}

\chapter{Abstract}

Komprimierter Inhalt der Arbeit (Kurzfassung) in Englisch.\\
Länge maximal 1 Seite

\autocite[S.24]{ederer2024}


\microtypesetup{protrusion=false}


\mainmatter{} % turns on chapter numbering, resets page numbering and uses arabic numerals for page numbers


\appendix
\renewcommand{\thechapter}{\Alph{chapter}}
\renewcommand{\thesection}{\Alph{section}}

\phantomsection
\addcontentsline{toc}{chapter}{Anhang}
\chapter*{Anhang}
Im Anhang sind all jene Informationen zu finden, die den Lesefluss der Arbeit stören würden, doch für die 
Vollständigkeit der Arbeit notwendig sind, z.B. Datentabellen, Fragebögen, etc.
Der Anhang unterliegt nicht mehr der Gliederung, sondern wird mit Anhang A, Anhang B, … gekennzeichnet.


\chapter{Theoretischer Hintergrund und Stand der Forschung} \label{cha:fundaments}
Die in dieser Formatvorlage beispielhaft enthaltenen Überschriften sind auf die im konkreten Fall tatsächlich passenden Überschriften anzupassen.\\
In diesem Teil der Arbeit werden die zum eindeutigen Verständnis unbedingt erforderlichen 
Grundlagen und Definitionen sowie die Erklärung wichtiger Begriffe angeführt.\\
Die Gliederungspunkte müssen möglichst prägnant bezeichnet werden.



\section{Stand des Wissens/Forschung/Technik} \label{cha:hydrostaticPressure}
Die neuesten Entwicklungen und Arbeiten auf diesem Gebiet (Stand der Wissenschaft oder auch state-of-the-art) 
sind darzulegen, wobei diese je nach Thema auch in der 3. Gliederungsebene behandelt werden können.\\
Dieses Unterkapitel inkl. eventueller Unterkapitel muss enthalten sein.



\section{Weitere frei zu wählende Unterkapitel} \label{cha:generalDefinitions}


\section{Zwischenfazit} \label{cha:zwischenfazit}
Alternativ kann dieses Unterkapitel auch „Interpretation in Bezug auf die Forschungsfrage und Literatur“ genannt werden.
Sie fassen die Erkenntnisse kurz zusammen und beantworten bzw. interpretieren diese im Sinne der Forschungsfrage.
Dieses Unterkapitel muss enthalten sein.





\appendix
\renewcommand{\thechapter}{\Alph{chapter}}
\renewcommand{\thesection}{\Alph{section}}

\phantomsection
\addcontentsline{toc}{chapter}{Anhang}
\chapter*{Anhang}
Im Anhang sind all jene Informationen zu finden, die den Lesefluss der Arbeit stören würden, doch für die 
Vollständigkeit der Arbeit notwendig sind, z.B. Datentabellen, Fragebögen, etc.
Der Anhang unterliegt nicht mehr der Gliederung, sondern wird mit Anhang A, Anhang B, … gekennzeichnet.


\appendix
\renewcommand{\thechapter}{\Alph{chapter}}
\renewcommand{\thesection}{\Alph{section}}

\phantomsection
\addcontentsline{toc}{chapter}{Anhang}
\chapter*{Anhang}
Im Anhang sind all jene Informationen zu finden, die den Lesefluss der Arbeit stören würden, doch für die 
Vollständigkeit der Arbeit notwendig sind, z.B. Datentabellen, Fragebögen, etc.
Der Anhang unterliegt nicht mehr der Gliederung, sondern wird mit Anhang A, Anhang B, … gekennzeichnet.


\appendix
\renewcommand{\thechapter}{\Alph{chapter}}
\renewcommand{\thesection}{\Alph{section}}

\phantomsection
\addcontentsline{toc}{chapter}{Anhang}
\chapter*{Anhang}
Im Anhang sind all jene Informationen zu finden, die den Lesefluss der Arbeit stören würden, doch für die 
Vollständigkeit der Arbeit notwendig sind, z.B. Datentabellen, Fragebögen, etc.
Der Anhang unterliegt nicht mehr der Gliederung, sondern wird mit Anhang A, Anhang B, … gekennzeichnet.


\appendix
\renewcommand{\thechapter}{\Alph{chapter}}
\renewcommand{\thesection}{\Alph{section}}

\phantomsection
\addcontentsline{toc}{chapter}{Anhang}
\chapter*{Anhang}
Im Anhang sind all jene Informationen zu finden, die den Lesefluss der Arbeit stören würden, doch für die 
Vollständigkeit der Arbeit notwendig sind, z.B. Datentabellen, Fragebögen, etc.
Der Anhang unterliegt nicht mehr der Gliederung, sondern wird mit Anhang A, Anhang B, … gekennzeichnet.

%\nocite{*} % includes complete bibliography, including entries that remain uncited, RMF
\printbibliography[heading=bibnumbered, title=\literatureLabel] % bibliography


% Use an optional list of tables / figures / algorithms / listings(code).

\renewcommand{\listfigurename}{Abbildungsverzeichnis}
\listoffigures
\clearpage

\renewcommand{\listtablename}{Tabellenverzeichnis}
\listoftables
\clearpage

%\lstlistoflistings
%\cleardoublepage{}
%\listofalgorithms
%\addcontentsline{toc}{chapter}{List of Algorithms} 
%\cleardoublepage{}
%\cleardoublepage{}
%\microtypesetup{protrusion=true}

%\printglossary[type=\acronymtype]
%\printglossary

%\printunsrtglossaries

%\printunsrtglossary[type=\acronymtype, style=tab_style, nonumberlist=true, title={Acronyms}]
%\printunsrtglossary[type=entry, style=tab_style_sym, title={Definitions}]

%\cleardoublepage{}
%\pagebreak

%add your end matter here
%\{} % resets chapter numbering, uses letters for chapter numbers and doesn't fiddle with page numbering
%\backmatter{} %turns off chapter numbering and doesn't fiddle with page numberin
%\input{C_BackMatter/glossary}
\input{Formelverzeichnis.tex}
\input{acronyms}
\clearpage


\appendix
\renewcommand{\thechapter}{\Alph{chapter}}
\renewcommand{\thesection}{\Alph{section}}

\phantomsection
\addcontentsline{toc}{chapter}{Anhang}
\chapter*{Anhang}
Im Anhang sind all jene Informationen zu finden, die den Lesefluss der Arbeit stören würden, doch für die 
Vollständigkeit der Arbeit notwendig sind, z.B. Datentabellen, Fragebögen, etc.
Der Anhang unterliegt nicht mehr der Gliederung, sondern wird mit Anhang A, Anhang B, … gekennzeichnet.
\clearpage
\phantomsection
\addcontentsline{toc}{chapter}{Tooldokumentation}
\chapter*{Tooldokumentation}
Hier sind alle genutzten (KI) Tools zu dokumentieren, die über die im Methodenkapitel eingesetzten Tools 
hinausgehen.\\
\begin{table}[h]
    \centering
    \renewcommand{\arraystretch}{1.2}
    \begin{tabular}{|c|c|c|c|}
        \hline
        \textbf{(KI) Tool} & \textbf{Anwendungsbereich} & \textbf{Betroffener Teil} & \textbf{Anmerkungen} \\
        \hline
        text  & text & text & text \\
        text & text & text& text \\
        \hline
    \end{tabular}
\end{table}
\input{C_BackMatter/Eidesstattliche Erklärung.tex}

%\input{C_BackMatter/Anhang}
\end{document}